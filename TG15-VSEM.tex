\documentclass[]{article}
\usepackage{lmodern}
\usepackage{amssymb,amsmath}
\usepackage{ifxetex,ifluatex}
\usepackage{fixltx2e} % provides \textsubscript
\ifnum 0\ifxetex 1\fi\ifluatex 1\fi=0 % if pdftex
  \usepackage[T1]{fontenc}
  \usepackage[utf8]{inputenc}
\else % if luatex or xelatex
  \ifxetex
    \usepackage{mathspec}
  \else
    \usepackage{fontspec}
  \fi
  \defaultfontfeatures{Ligatures=TeX,Scale=MatchLowercase}
\fi
% use upquote if available, for straight quotes in verbatim environments
\IfFileExists{upquote.sty}{\usepackage{upquote}}{}
% use microtype if available
\IfFileExists{microtype.sty}{%
\usepackage{microtype}
\UseMicrotypeSet[protrusion]{basicmath} % disable protrusion for tt fonts
}{}
\usepackage[margin=1in]{geometry}
\usepackage{hyperref}
\hypersetup{unicode=true,
            pdftitle={TG15 manuscript},
            pdfauthor={David Cameron, Mike Dietze and Marcel van Oijen},
            pdfborder={0 0 0},
            breaklinks=true}
\urlstyle{same}  % don't use monospace font for urls
\usepackage{graphicx,grffile}
\makeatletter
\def\maxwidth{\ifdim\Gin@nat@width>\linewidth\linewidth\else\Gin@nat@width\fi}
\def\maxheight{\ifdim\Gin@nat@height>\textheight\textheight\else\Gin@nat@height\fi}
\makeatother
% Scale images if necessary, so that they will not overflow the page
% margins by default, and it is still possible to overwrite the defaults
% using explicit options in \includegraphics[width, height, ...]{}
\setkeys{Gin}{width=\maxwidth,height=\maxheight,keepaspectratio}
\IfFileExists{parskip.sty}{%
\usepackage{parskip}
}{% else
\setlength{\parindent}{0pt}
\setlength{\parskip}{6pt plus 2pt minus 1pt}
}
\setlength{\emergencystretch}{3em}  % prevent overfull lines
\providecommand{\tightlist}{%
  \setlength{\itemsep}{0pt}\setlength{\parskip}{0pt}}
\setcounter{secnumdepth}{5}
% Redefines (sub)paragraphs to behave more like sections
\ifx\paragraph\undefined\else
\let\oldparagraph\paragraph
\renewcommand{\paragraph}[1]{\oldparagraph{#1}\mbox{}}
\fi
\ifx\subparagraph\undefined\else
\let\oldsubparagraph\subparagraph
\renewcommand{\subparagraph}[1]{\oldsubparagraph{#1}\mbox{}}
\fi

%%% Use protect on footnotes to avoid problems with footnotes in titles
\let\rmarkdownfootnote\footnote%
\def\footnote{\protect\rmarkdownfootnote}

%%% Change title format to be more compact
\usepackage{titling}

% Create subtitle command for use in maketitle
\newcommand{\subtitle}[1]{
  \posttitle{
    \begin{center}\large#1\end{center}
    }
}

\setlength{\droptitle}{-2em}
  \title{TG15 manuscript}
  \pretitle{\vspace{\droptitle}\centering\huge}
  \posttitle{\par}
  \author{David Cameron, Mike Dietze and Marcel van Oijen}
  \preauthor{\centering\large\emph}
  \postauthor{\par}
  \predate{\centering\large\emph}
  \postdate{\par}
  \date{2018-09-25}


\begin{document}
\maketitle

{
\setcounter{tocdepth}{2}
\tableofcontents
}
\section{Introduction}\label{introduction}

Main issue

\begin{itemize}
\tightlist
\item
  Because of the difference in cost in collecting automatic measurements
  such as eddy covariance data and manual measurements such as soil and
  plant carbon stocks, there is often two orders of magnitude difference
  or more imbalance between the number of measurements available from
  different parts of the forest ecosystem.
\item
  Since each measurement point normally counts as an independent piece
  of data in the Likelihood of Bayesian calibration the influence of the
  sparse observations can often be overwhelmed by the higher frequency
  data (Cameron et al (in review 2018).
\item
  As more and more EO data becomes available (e.g.~Sentinal) this issue
  of extremely imbalanced datasets is likely to worsen significantly.
\end{itemize}

What is the effective information content (IC) of observations (possibly
include or is this really a different issue? - discuss Mike/Marcel)

\begin{itemize}
\tightlist
\item
  One eddy covariance tower with 17000 measurements count this as n = 1
  or n = 17000 obs?
\item
  Aggregate 5 min data to 10 min retain essentially same IC but sample
  size halved!
\item
  Spectral analysis of eddy covariance NEE data two peaks:
  annual(seasonal) and diurnal
\item
  Often assume in BC that each data*point provides independent
  information
\item
  If biases in data or model errors are not independent
\end{itemize}

To-date solutions generally ad hoc and/or arbitrary

\begin{itemize}
\tightlist
\item
  Literature review\ldots{}
\item
  Ignore but then models over-fitted and uncertainty underestimated
\item
  Apply arbitrary weights to rebalance influence of data in BC
\item
  Thin the number of eddy covariance obs

  \begin{itemize}
  \tightlist
  \item
    throwing away useful information
  \end{itemize}
\end{itemize}

Purpose of paper:

\begin{enumerate}
\def\labelenumi{\arabic{enumi}.}
\tightlist
\item
  To identify the issue of using unbalanced data in parameter
  calibration of ecosystem models using artificial experiments.
\item
  Develop a general methodology for identifying whether and where the
  issue becomes a problem.
\item
  Start to explore the simplest modifications to the likelihood to
  represent model structural error and data bias to help ameliorate the
  issue identified in (1).
\end{enumerate}

Artificial experiments:

\begin{itemize}
\tightlist
\item
  Developed a very simple process-besed ecosystem model (VSEM) as a
  testbed for experiments. Simple enough that results are easily
  understood. Sufficiently complex that we can be confident that the
  model/data issues identified here would also been seen in more
  complex/realistic process-based ecosystem models.
\item
  Calibration data model's own output to control:

  \begin{enumerate}
  \def\labelenumi{\arabic{enumi}.}
  \tightlist
  \item
    test influence of model and data perfection/imperfection
  \item
    test influence of balance/unbalance of data in the calibration
  \end{enumerate}
\end{itemize}

\section{Methods}\label{methods}

\subsection{VSEM model}\label{vsem-model}

\begin{itemize}
\tightlist
\item
  very simple toy model. not realistic but simplified form with and
  similar structure to many ecosystem models.
\end{itemize}

Photosynthesis equation

\begin{align}
NPP &= PAR \times LUE \times (1 - \exp^{(-KEXT \times C_v)})\\
\end{align}

\begin{itemize}
\tightlist
\item
  PAR Photosynthetically active radiation
\item
  LUE Light use efficiency of NPP (Ra implicit)
\item
  KEXT Beer's law light extinction coeff
\item
  \(C_v\) Vegetation carbon
\end{itemize}

Carbon pool state equations

\begin{align}
\frac{dC_v}{dt}  &= A_v \times NPP &- \frac{C_v}{\tau_v} \\
\frac{dC_r}{dt}  &= (1.0-A_v) \times NPP &- \frac{C_r}{\tau_r}\\
\frac{dC_s}{dt}  &= \frac{C_r}{\tau_r} + \frac{C_v}{\tau_v} &- \frac{C_s}{\tau_s}
\end{align}

\begin{itemize}
\tightlist
\item
  \(C_v\), \(C_r\) and \(C_s\) : Carbon in vegetation, root and soil
  pools
\end{itemize}

\subsection{Bayesian Calibration}\label{bayesian-calibration}

\begin{itemize}
\tightlist
\item
  refer to forthcoming TG13 paper
\item
  R package used BayesianTools
\item
  DREAMzs algorithm
\end{itemize}

\subsection{Idealised experiments with virtual data from
VSEM}\label{idealised-experiments-with-virtual-data-from-vsem}

\begin{itemize}
\tightlist
\item
  advantage we know what the truth is.
\item
  balanced versus unbalanced
\item
  perfect model and model with known error
\item
  data with and without known bias
\item
  include additive and multiplicative parameters to represent structural
  errors in the model and systematic biases in the data.
\end{itemize}

\section{Identifying the issue}\label{identifying-the-issue}

\subsection{Perfect model and balanced
data}\label{perfect-model-and-balanced-data}

\begin{itemize}
\tightlist
\item
  refer to Fig. (\ref{fig:perModbalDataPar}) and
  (\ref{fig:perModbalDataOut})
\item
  `true' parameters largely found in posterior
\item
  posterior parameter controlling data error close to 0.1 coefficient of
  variance originally imposed to create observations from `true' model
  output.
\item
  50\% quantile red line very close to green `truth' line
\item
  posterior very narrow marked by 2.5\% and 97.5\% quantiles
\item
  most data within predictive interval
\end{itemize}

\begin{figure}
\centering
\includegraphics{TG15-VSEM_files/figure-latex/unnamed-chunk-10-1.pdf}
\caption{\label{fig:perModbalDataPar}Perfect model, balanced data (NEE,
Cv, Cs: 2048 obs). Marginal posterior distribution of model parameters
and intital states. The red line marks the `true' parameter values.}
\end{figure}

\begin{figure}
\centering
\includegraphics{TG15-VSEM_files/figure-latex/unnamed-chunk-11-1.pdf}
\caption{\label{fig:perModbalDataOut}Perfect model, balanced data (NEE,
Cv, Cs: 2048 obs). Observations included in the calibration marked with
a `+'. Red line 50\% quantile posterior distribution. Green line is the
`true' model output. Dark brown shading 2.5\% 97.5\% quantile posterior
distribution. Light brown shading 2.5\% 97.5\% predictive interval.}
\end{figure}

\subsection{Perfect model and unbalanced
data}\label{perfect-model-and-unbalanced-data}

\begin{itemize}
\tightlist
\item
  refer to Fig. (\ref{fig:perModunbalDataPar}) and
  (\ref{fig:perModunbalDataOut})
\item
  despite unbalanced data

  \begin{itemize}
  \tightlist
  \item
    parameter values still close to true values in posterior
  \item
    model output still close to truth even for vegetative carbon
  \item
    although now greater uncertainty for Cv then for balanced
    calibration
  \end{itemize}
\end{itemize}

\begin{figure}
\centering
\includegraphics{TG15-VSEM_files/figure-latex/unnamed-chunk-13-1.pdf}
\caption{\label{fig:perModunbalDataPar}Perfect model, unbalanced data
(NEE, Cs: 2048 obs, Cv: 6 obs). Marginal posterior distribution of model
parameters and intital states. The red line marks the `true' parameter
values.}
\end{figure}

\begin{figure}
\centering
\includegraphics{TG15-VSEM_files/figure-latex/unnamed-chunk-14-1.pdf}
\caption{\label{fig:perModunbalDataOut}Perfect model, unbalanced data
(NEE, Cs: 2048 obs, Cv: 6 obs). Observations included in the calibration
marked with a `+'. Red line 50\% quantile posterior distribution. Green
line is the `true' model output. Dark brown shading 2.5\% 97.5\%
quantile posterior distribution. Light brown shading 2.5\% 97.5\%
predictive interval.}
\end{figure}

\subsection{Model with error and balanced
data}\label{model-with-error-and-balanced-data}

\begin{itemize}
\tightlist
\item
  refer to Fig. (\ref{fig:errModbalDataPar}) and
  (\ref{fig:errModbalDataOut})
\item
  Root pool is effectively removed from the model by initialising pool
  to zero and setting allocation to roots to zero. The loss of the root
  pool has introduced a significant structural error to the model.
\item
  Parameter posteriors now quite far away from `true' values.

  \begin{itemize}
  \tightlist
  \item
    Especially parameter which controls turnover of vegetation so that
    rate of turnover to soil is now more than doubled.
  \end{itemize}
\item
  Parameters calibration seems to have somewhat `absorbed' the model
  structural error so that

  \begin{itemize}
  \tightlist
  \item
    outputs where there was data included in the BC are still not too
    far away from the `truth' line.
  \item
    Cv now has too much variability but average increase not too bad.
  \item
    most data still within predictive interval.
  \end{itemize}
\end{itemize}

\begin{figure}
\centering
\includegraphics{TG15-VSEM_files/figure-latex/unnamed-chunk-16-1.pdf}
\caption{\label{fig:errModbalDataPar}Model with error, balanced data.
Marginal posterior distribution of model parameters and intital states.
The red line marks the `true' parameter values.}
\end{figure}

\begin{figure}
\centering
\includegraphics{TG15-VSEM_files/figure-latex/unnamed-chunk-17-1.pdf}
\caption{\label{fig:errModbalDataOut}Model with error, balanced data.
Observations included in the calibration marked with a `+'. Red line
50\% quantile posterior distribution. Green line is the `true' model
output. Dark brown shading 2.5\% 97.5\% quantile posterior distribution.
Light brown shading 2.5\% 97.5\% predictive interval.}
\end{figure}

\subsection{Model with error and unbalanced
data}\label{model-with-error-and-unbalanced-data}

\begin{itemize}
\tightlist
\item
  refer to Fig. (\ref{fig:errModunbalDataPar}) and
  (\ref{fig:errModunbalDataOut})
\item
  KEXT smaller, LUE larger, Cv larger, tauS smaller, tauV much larger,
  Cs much larger

  \begin{itemize}
  \tightlist
  \item
    Generally parameters closer to their `true' value. Less `absorbing'
    of the model structural error.
  \end{itemize}
\item
  NEE and soil carbon pools look largely unchanged to balanced run and
  close to data and predictive interval.
\item
  significant change to soil vegetation pool

  \begin{itemize}
  \tightlist
  \item
    data outside of posterior and close to one edge of predictive
    interval
  \item
    departure from `truth' line growing in time
  \end{itemize}
\item
  General sense is that six vegetation data points are being somewhat
  ignored by the calibration in favour of the more plentiful NEE and
  soil carbon data.
\end{itemize}

\begin{figure}
\centering
\includegraphics{TG15-VSEM_files/figure-latex/unnamed-chunk-19-1.pdf}
\caption{\label{fig:errModunbalDataPar}Model with error, unbalanced data
(NEE, Cs: 2048 obs, Cv: 6 obs). Marginal posterior distribution of model
parameters and intital states. The red line marks the `true' parameter
values.}
\end{figure}

\begin{figure}
\centering
\includegraphics{TG15-VSEM_files/figure-latex/unnamed-chunk-20-1.pdf}
\caption{\label{fig:errModunbalDataOut}Model with error, unbalanced data
(NEE, Cs: 2048 obs, Cv: 6 obs). Observations included in the calibration
marked with a `+'. Red line 50\% quantile posterior distribution. Green
line is the `true' model output. Dark brown shading 2.5\% 97.5\%
quantile posterior distribution. Light brown shading 2.5\% 97.5\%
predictive interval.}
\end{figure}

\subsection{Perfect model and balanced data with a multiplicative
bias}\label{perfect-model-and-balanced-data-with-a-multiplicative-bias}

\begin{itemize}
\tightlist
\item
  refer to Fig. (\ref{fig:perModbalDataBiasPar}) and
  (\ref{fig:perModbalDataBiasOut})
\item
  soil carbon pool data multiplied by two to represent data that has a
  systematic bias
\item
  Parameters KEXT larger, tauV smaller, tauS much smaller, Cs much
  larger
\item
  as for model error parameter calibration `absorb' the error

  \begin{itemize}
  \tightlist
  \item
    initial value of soil pool parameter more than double `true' value
  \item
    parameter controlling turnover time of soil approximately doubled to
    keep soil carbon pool high.
  \end{itemize}
\item
  calibrated outputs again reasonably close to observations for NEE and
  Cv

  \begin{itemize}
  \tightlist
  \item
    slope of soil carbon pool with time too shallow (so some data
    outside predictive interval) but compares well with `true' line
    (influence of other more accurate NEE and vegetative carbon data?)
  \end{itemize}
\item
  data and calibrated output of soil carbon greater than `true' line by
  a factor of two as might be expected to match calibration data.
\end{itemize}

\begin{figure}
\centering
\includegraphics{TG15-VSEM_files/figure-latex/unnamed-chunk-22-1.pdf}
\caption{\label{fig:perModbalDataBiasPar}Perfect model and balanced data
with a multiplicative bias. Marginal posterior distribution of model
parameters and intital states. The red line marks the `true' parameter
values.}
\end{figure}

\begin{figure}
\centering
\includegraphics{TG15-VSEM_files/figure-latex/unnamed-chunk-23-1.pdf}
\caption{\label{fig:perModbalDataBiasOut}Perfect model and balanced data
with a multiplicative bias. Observations included in the calibration
marked with a `+'. Red line 50\% quantile posterior distribution. Green
line is the `true' model output. Dark brown shading 2.5\% 97.5\%
quantile posterior distribution. Light brown shading 2.5\% 97.5\%
predictive interval.}
\end{figure}

\subsection{Perfect model and unbalanced data with a multiplicative
bias}\label{perfect-model-and-unbalanced-data-with-a-multiplicative-bias}

\begin{itemize}
\tightlist
\item
  refer to Fig. (\ref{fig:perModunbalDataBiasPar}) and
  (\ref{fig:perModunbalDataBiasOut})
\item
  most parameters are far away from their `true' values
\item
  soil carbon improved fit to data versus previous balanced data
  calibration
\item
  six vegetative carbon data points effectively ignored and overpowered
  by NEE and soil carbon data in the calibration.
\end{itemize}

\begin{figure}
\centering
\includegraphics{TG15-VSEM_files/figure-latex/unnamed-chunk-25-1.pdf}
\caption{\label{fig:perModunbalDataBiasPar}Perfect model and unbalanced
data with a multiplicative bias. Marginal posterior distribution of
model parameters and intital states. The red line marks the `true'
parameter values.}
\end{figure}

\begin{figure}
\centering
\includegraphics{TG15-VSEM_files/figure-latex/unnamed-chunk-26-1.pdf}
\caption{\label{fig:perModunbalDataBiasOut}Perfect model and unbalanced
data with a multiplicative bias. Observations included in the
calibration marked with a `+'. Red line 50\% quantile posterior
distribution. Green line is the `true' model output. Dark brown shading
2.5\% 97.5\% quantile posterior distribution. Light brown shading 2.5\%
97.5\% predictive interval.}
\end{figure}

\subsection{Model with error and unbalanced data with a multiplicative
bias}\label{model-with-error-and-unbalanced-data-with-a-multiplicative-bias}

\begin{itemize}
\tightlist
\item
  refer to Fig. (\ref{fig:errModunbalDataBiasPar}) and
  (\ref{fig:errModunbalDataBiasOut})
\item
  similar to calibration with bias in data above.
\item
  vegetative carbon perhaps very slightly closer to the data than when
  only data bias was present. This may indicate compensating errors in
  the model and data.
\end{itemize}

\begin{figure}
\centering
\includegraphics{TG15-VSEM_files/figure-latex/unnamed-chunk-28-1.pdf}
\caption{\label{fig:errModunbalDataBiasPar}Model with error and
unbalanced data with a multiplicative bias. Marginal posterior
distribution of model parameters and intital states. The red line marks
the `true' parameter values.}
\end{figure}

\begin{figure}
\centering
\includegraphics{TG15-VSEM_files/figure-latex/unnamed-chunk-29-1.pdf}
\caption{\label{fig:errModunbalDataBiasOut}Model with error and
unbalanced data with a multiplicative bias. Observations included in the
calibration marked with a `+'. Red line 50\% quantile posterior
distribution. Green line is the `true' model output. Dark brown shading
2.5\% 97.5\% quantile posterior distribution. Light brown shading 2.5\%
97.5\% predictive interval.}
\end{figure}

\section{Diagnosing the issue}\label{diagnosing-the-issue}

\subsection{Comparing model output with virtual data as
truth.}\label{comparing-model-output-with-virtual-data-as-truth.}

Moving on from identifying the issue in the previous section, here we
develop a tool for helping to diagnose at what point and to what extent
having unbalanced data in Bayesian calibration (BC) becomes an issue
when models and data are imperfect.

This is done by running a number of calibrations with perfect and
imperfect models where the quantity and imbalance of data used increases
with each calibration. Here we chose an increasing power series of two
(2\textsuperscript{3},2\textsuperscript{4} \ldots{}
2\textsuperscript{11}) for the increase in the quantity of calibration
data; eight calibrations in all. In the balanced data BC case,
quantities of NEE, vegetative carbon and soil carbon data included in
the BC all increased in tandem in each subsequent calibration. For the
unbalanced BC case, NEE and soil carbon data increased as before but the
quantity of vegetative carbon data included in the BC was held fixed at
six data points for each of the eight calibrations. After running the
calibrations the VSEM was rerun with the maximum a posteriori (MAP)
vector and the RMS difference with the `true' data was calculated and
plotted (Fig. \ref{fig:rmsMAPTruth}).

The figure shows broad similarity in results except for vegetative
carbon case when the model has an error and where there is an imbalanced
in calibration data. In general, the RMS difference has a tendency to go
down as the quantity of data included in calibration increases. There is
also a marked grouping of results with the perfect model getting closer
to the data than the model with the error, as might be expected. For NEE
and soil carbon with an imperfect model, the unbalanced calibration gets
closer to the data than the balanced calibration especially as the
quantity of calibration data increases. This is in marked contrast to
vegetative carbon where RMS differences increase significantly as
quantity of calibration data increases when the model has an error and
when there is an imbalanced in calibration data. This increase in RMS
difference for vegetative carbon occurs in tandem with the decreases
noted already from NEE and soil carbon. This signature of increasing RMS
difference for the low quantity data output versus the decreasing RMS
difference for the high quantity can be used to diagnose when large
imbalances in calibrations data with imperfect models and data start to
become an issue. In this case, it appears after the quantity of data
included in the calibration exceeds 32 but this will be different for
each model, likelihood function and for each dataset used in
calibrations.

\begin{figure}
\centering
\includegraphics{TG15-VSEM_files/figure-latex/unnamed-chunk-31-1.pdf}
\caption{\label{fig:rmsMAPTruth}Each point in the three graphs (NEE,
vegetative carbon, and soil carbon) represents the RMS difference
between the VSEM model and the `truth' run with different maximum a
posteriori (MAP) vectors. The MAP vector at each point is obtained from
a Bayesian calibration (BC) where the quantity of data included in the
BC increases in a sequence along the x-axis following the exponentiation
of base two. For the balanced calibration case (red and cyan) vegetative
carbon data increases in tandem with NEE and soil carbon. For the
unbalanced calibration case (green and purple) the quantity of
vegetative carbon data is held fixed at six data values for each point
along the x-axis. The VSEM model is either `perfect' (cyan and purple)
or has a known error (red and green) relative to the `true' data that
was derived from it.}
\end{figure}

\subsection{\texorpdfstring{Comparing model output against
``obervations''}{Comparing model output against obervations}}\label{comparing-model-output-against-obervations}

The diagnosis made in the previous section had the benefit of access to
the `true' data and a perfect model. Unfortunately this is never the
case for real world ecological model calibrations. Therefore, here we
have repeated the previous graph Fig.(\ref{fig:rmsMAPTruth}) with just
the imperfect model and the imbalanced calibration, but with RMS
differences now calculated against observations (NEE: 2048 points,
vegetative carbon: 6 points, soil carbon: 2048 points) (Fig.
\ref{fig:rmsMAPObs}). While there are clear differences in the RMS
values versus the previous graph, as might be expected, the broad-scale
signature of increasing RMS difference for vegetative carbon and
decreasing RMS difference for NEE and soil carbon is retained. As
before, this graph can be used to diagnose when the imbalanced in data
is starting to interact with the erroneous model. In this case, as
before, this occurs for a data quantity greater than 32.

\begin{figure}
\centering
\includegraphics{TG15-VSEM_files/figure-latex/unnamed-chunk-33-1.pdf}
\caption{\label{fig:rmsMAPObs}Each point in the three graphs (NEE,
vegetative carbon, and soil carbon) represents the RMS difference
between the VSEM model and virtual observations run with different
maximum a posteriori (MAP) vectors. The MAP vector at each point is
obtained from a Bayesian calibration (BC) where the quantity of data
included in the BC for NEE and soil carbon increases in a sequence along
the x-axis following the exponentiation of base two. The quantity of
vegetaive carbon data is held fixed at six for all points in the graphs.
The VSEM model used has a known error relative to the virtual
observations that was derived from it.}
\end{figure}

\section{Changes to the Likelihood to represent model and data
errors}\label{changes-to-the-likelihood-to-represent-model-and-data-errors}

\subsection{Model with error and unbalanced perfect data with additive
and multiplicative parameters to represent model
error.}\label{model-with-error-and-unbalanced-perfect-data-with-additive-and-multiplicative-parameters-to-represent-model-error.}

\begin{itemize}
\tightlist
\item
  refer to Fig. (\ref{fig:errModunbalDatamodLikePar}) and
  (\ref{fig:errModunbalDatamodLikeOut})
\item
  KEXT, LUE, Cv, Cs and error-coeffVar are now significantly closer to
  the `true' values. tauS is not but is much more uncertain.
\item
  vegetative carbon much improved

  \begin{itemize}
  \tightlist
  \item
    5 out of 6 data points are now inside the posterior confidence
    interval
  \item
    50\% quantile line now much closer to the `true' line.
  \end{itemize}
\end{itemize}

\begin{figure}
\centering
\includegraphics{TG15-VSEM_files/figure-latex/unnamed-chunk-35-1.pdf}
\caption{\label{fig:errModunbalDatamodLikePar}Model with error and
unbalanced data with additive and multiplicative parameters to represent
model error. Marginal posterior distribution of model parameters and
intital states. The red line marks the `true' parameter values.}
\end{figure}

\begin{figure}
\centering
\includegraphics{TG15-VSEM_files/figure-latex/unnamed-chunk-36-1.pdf}
\caption{\label{fig:errModunbalDatamodLikeOut}Model with error and
unbalanced data with additive and multiplicative parameters to represent
model error. Observations included in the calibration marked with a `+'.
Red line 50\% quantile posterior distribution. Green line is the `true'
model output. Dark brown shading 2.5\% 97.5\% quantile posterior
distribution. Light brown shading 2.5\% 97.5\% predictive interval.}
\end{figure}

\subsection{Perfect model and and unbalanced data with a multiplicative
bias and additive and multiplicative parameters to represent the
bias.}\label{perfect-model-and-and-unbalanced-data-with-a-multiplicative-bias-and-additive-and-multiplicative-parameters-to-represent-the-bias.}

\begin{itemize}
\tightlist
\item
  refer to Fig. (\ref{fig:perModunbalDataBiasmodLikePar}) and
  (\ref{fig:perModunbalDataBiasmodLikeOut})
\item
  many model parameters (KEXT, LUE, tauV, tauS, initial Cv) much closer
  to `true' values
\item
  multiplicative bias multiplication parameter modmultCs centred around
  2.25 which is close to the multiplication factor applied to the data.
\item
  NEE and soil carbon close to the data and within the predictive
  interval.
\item
  vegetative carbon pool much improved with all data points covered by
  the posterior credible interval.
\end{itemize}

\begin{figure}
\centering
\includegraphics{TG15-VSEM_files/figure-latex/unnamed-chunk-38-1.pdf}
\caption{\label{fig:perModunbalDataBiasmodLikePar}Perfect model and and
unbalanced data with a multiplicative bias and additive and
multiplicative parameters to represent the bias. Marginal posterior
distribution of model parameters and intital states. The red line marks
the `true' parameter values.}
\end{figure}

\begin{figure}
\centering
\includegraphics{TG15-VSEM_files/figure-latex/unnamed-chunk-39-1.pdf}
\caption{\label{fig:perModunbalDataBiasmodLikeOut}Perfect model and and
unbalanced data with a multiplicative bias and additive and
multiplicative parameters to represent the bias. Observations included
in the calibration marked with a `+'. Red line 50\% quantile posterior
distribution. Green line is the `true' model output. Dark brown shading
2.5\% 97.5\% quantile posterior distribution. Light brown shading 2.5\%
97.5\% predictive interval.}
\end{figure}

\subsection{Model with error and and unbalanced data with a
multiplicative bias and additive and multiplicative parameters to
represent model error and the data
bias.}\label{model-with-error-and-and-unbalanced-data-with-a-multiplicative-bias-and-additive-and-multiplicative-parameters-to-represent-model-error-and-the-data-bias.}

\begin{figure}
\centering
\includegraphics{TG15-VSEM_files/figure-latex/unnamed-chunk-41-1.pdf}
\caption{\label{fig:errModunbalDataBiasmodLikePar}Model with error and
and unbalanced data with a multiplicative bias and additive and
multiplicative parameters to represent model error and the data bias.
Marginal posterior distribution of model parameters and intital states.
The red line marks the `true' parameter values.}
\end{figure}

\begin{figure}
\centering
\includegraphics{TG15-VSEM_files/figure-latex/unnamed-chunk-42-1.pdf}
\caption{\label{fig:errModunbalDataBiasmodLikeOut}Model with error and
and unbalanced data with a multiplicative bias and additive and
multiplicative parameters to represent model error and the data bias.
Observations included in the calibration marked with a `+'. Red line
50\% quantile posterior distribution. Green line is the `true' model
output. Dark brown shading 2.5\% 97.5\% quantile posterior distribution.
Light brown shading 2.5\% 97.5\% predictive interval.}
\end{figure}

\begin{itemize}
\tightlist
\item
  refer to Fig. (\ref{fig:errModunbalDataBiasmodLikePar}) and
  (\ref{fig:errModunbalDataBiasmodLikeOut})
\item
  as above many parameters improved (KEXT, LUE, tauS, initial Cv)
\item
  modmultCs value centered \textasciitilde{}1.8 compromise between value
  in Fig. (\ref{fig:perModunbalDataBiasmodLikePar}) for bias only and
  Fig. (\ref{fig:errModunbalDatamodLikePar}) model error only.
\item
  vegetative pool much improved with

  \begin{itemize}
  \tightlist
  \item
    5 out of 6 data points within posterior credible interval
  \item
    50\% quantile line now much closer to the `true' line.
  \item
    similar to Fig. (\ref{fig:errModunbalDatamodLikeOut})
  \end{itemize}
\end{itemize}

\section{Discussion}\label{discussion}

\subsection{Identifying the issue with unbalanced dataset
BC}\label{identifying-the-issue-with-unbalanced-dataset-bc}

\begin{itemize}
\tightlist
\item
  Unbalanced data are not an issue though uncertainty is larger.
\item
  For a model with a significant structural error or systematic bias in
  the calibration data parameters `absorb' the error so that model
  output is not too far away from the data.
\item
  With a model structural error or data with a systematic bias general
  sense is that sparse data are somewhat ignored in the BC in favour of
  the plentiful data.
\item
  This is what we often observe with unbalanced datasets in BC. For
  example, Cameron et al (2018) but results here make it apparent that
  the issue is the presence of the structural error in the model rather
  than that the calibration data are unbalanced.
\end{itemize}

\subsection{Diagnostic tool
introduced}\label{diagnostic-tool-introduced}

\begin{itemize}
\tightlist
\item
  Have created a methodology and graph which can be used in many
  applications to diagnose and help judge the severity of the issue.
\end{itemize}

\subsection{Representing model and data error in BC helps to alleviate
the
issue}\label{representing-model-and-data-error-in-bc-helps-to-alleviate-the-issue}

\begin{itemize}
\tightlist
\item
  In this very simple example we were able to demonstrate a significant
  improvement by including terms in the likelihood to represent model
  and data error.
\item
  In more real-world applications representing model and data error in
  BC will be much more challenging but the analysis demonstrated here
  shows how to deal with the issue of calibrating with unbalanced
  datasets without resorting to ad hoc methods.
\end{itemize}


\end{document}
